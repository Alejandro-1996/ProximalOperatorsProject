\section{Conclusion} \label{sect:Conclusion}
In this project I have presented a thorough study of the combination of the proximal operator of common image regularizers with nonnegativity constraints. This is of interest because current algorithms of choice like ADMM perform variable splitting, which has a heavy computational cost. Therefore a good combination can reduce the computational load of state-of-the-art image reconstruction techniques. 

Specifically, my experimental results suggest that all the family of norms $\|\cdot\|_p$ and $\|\cdot\|_p^p$ combine well with nonnegativity constraints, a result of interest in the domains of signal and image reconstructions. Furthermore, I found that group sparsity -or the $\|\cdot\|_{p, q}$ mixed-norm- also has good combination, as all studied cases comply with  \eqref{eq:prox_r(prox_nonneg)}. In the case of more complex image regularizers, I have found that nonisotropic $\operatorname{TV}$ combines well with nonnegativity constraints. Finally, I studied the Hessian-Schatten norm, and while the result was not compelling enough to jump to a conclusion, it is promising enough to further investigate the topic.

% Specifically, I showed -in an experimental setting- how all the family of norms $\ell_p$ and $\ell_p^p$ combine well with nonnegativity constraints, a result of interest in the domains of signal and image reconstructions. Furthermore, I found that group sparsity -or the $\ell_p, \ell_q$ mixed-norm- also has good combination, as all studied cases comply with  \eqref{eq:prox_r(prox_nonneg)}. In the case of more complex image regularizers, I have found that non-isometric $\operatorname{TV}$ combines well with nonnegativity constraints. Finally, I studied the Hessian-Schatten norm, and while the result was not compelling enough to jump to a conclusion, it is promising enough to further investigate the topic. 

All of these cases are of interest because they can have a substantial effect on the performance of image reconstruction methods, and they are all commonly used regularizers. The next step is to prove the results found here using closed-form solutions, though it is out of the scope of this project. Moreover, as further work and before proving the results for closed form solutions, direct testing in an optimization library could be made. This would also allow testing on more varied settings, and images of different kinds.

\section{Acknowledgements} \label{sect:acknowledgements}
I would like to thank my supervisor, Dr. Pol del Aguila Pla, for constantly solving all kinds of doubts, from technical to theoretical. I also want to thank Pakshal Bohra, who provided me with MATLAB code to perform a similar $1$-dimensional $\operatorname{TV}$ experiment. Finally, from EPFL´s Biomedical Imaging Group, I would like to thank Thomas Debarre, and Prof. Michael Unser for fruitful discussions and feedback on the project.
